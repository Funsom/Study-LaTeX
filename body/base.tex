{\let\clearpage\relax \chapter{\LaTeX 基础}}

\section{命令}

%\begin{latexample}
%{\songti 爆竹声中一岁除,春风送暖入屠苏。}
%{\fangsong 家家乞巧望秋月,穿尽红丝几万条。}
%\end{latexample}

\subsection{命令形式}

命令的一般形式如下。

\begin{latex}{}
\命令名[可选参数1,可选参数2,...]{必要参数1,必要参数2,...}
\end{latex}

同时,根据命令的作用范围的不同,可大致把命令分为如下四种形式。

\section{文类}

\CTeX 提供四个中文文类,\texttt{ctexart、ctexrep、ctexbook和ctexbeamer}

\section{字体}

\begin{asparadesc}
	\item[等宽字体] Typewriter Family 英文的a和i在非等宽字体里面肯定宽度不一样,这样在大段文本里就不好辨认,等宽字体的所以字母宽度一样,笔画的起止还有装饰衬线(所以等宽字体多数属于衬线字体),易读性高
	\item[等线字体] 无字头字脚,笔画圆润,粗细均匀,例如Windows自带的Arial、黑体和幼圆
	\item[衬线字体] serif 在字的笔画开始、结束的地方有额外的装饰,而且笔画的粗细会有所不同,宋体就是一种最标准的serif字体
	\item[无衬线字体] sans serif 在字的笔画开始、结束的地方没有这些额外的装饰,而且笔画的粗细差不多
\end{asparadesc}

等宽字体一般用来书写代码,特别是使用缩进控制语法的\emph{python}语言,更需要等宽字体来书写代码了。

科学书写中文文档的第一步应该是调用\CTeX 宏包,其提供四种命令来调用在中文文档中常用的四种字体。

\begin{latex}{}
{\songti 爆竹声中一岁除,春风送暖入屠苏。}
{\fangsong 家家乞巧望秋月,穿尽红丝几万条。}
{\heiti 黄沙百战穿金甲,不破楼兰终不还。}
{\kaishu 君不见走马川行雪海边,平沙莽莽黄入天。}
\end{latex}

效果如下:

\begin{center}
	{\zihao{3}\songti 爆竹声中一岁除,春风送暖入屠苏。}\par
	{\zihao{3}\fangsong 家家乞巧望秋月,穿尽红丝几万条。}\par
	{\zihao{3}\heiti 黄沙百战穿金甲,不破楼兰终不还。}\par
	{\zihao{3}\kaishu 君不见走马川行雪海边,平沙莽莽黄入天。}
\end{center}

分全局和局部字体设置。
\subsection{全局字体设置}
中文的文档都要调用\emph{ctex}宏包,该宏包提供一个简单的参数可以设置全部正文的字体。

\begin{latex}{}
\setmainfont{Times New Roman}	%设置主字体,仅对西文起作用
\setCJKmainfont{SimSun}			%设置主字体,仅对中文起作用
\end{latex}

有时候需要改变\LaTeX 默认的等宽字体,如本文档的等宽字体设置。更改等宽字体之后,将会影响\verb|\texttt{text},\ttfamily,\tt|这些命令所作用的字体,还会影响默认使用等宽字体(如脚注,抄录环境)的环境。

\begin{latex}{}
\setmonofont{Source Code Pro}	%英文等宽
\setCJKmonofont{simfang.ttf}	%中文等宽,仿宋
\end{latex}

\subsection{局部字体设置}

\begin{latex}{}
\newfontfamily\daima{Consolas}	%使用\daima直接调用
\end{latex}

\subsection{在数学环境中使用中文}
默认情况下,数学环境中是不允许输入汉字的。当我们需要输入汉字作为变量的标识时,可以使用\verb|\text{要输入的汉字字符}|来完成这项工作。

\begin{codeshow}
$f_{\text{高温}}$
\end{codeshow}

\subsection{汉字“斜体”}
汉字没有加斜体。平常我们看到的加斜汉字,通常是几何变换得到的结果,非常的粗糙,并不严格满足排版要求;而真正的字形是需要精细的设计的。同时,汉字字体里面也很少有加粗体的设计。但是,有时候却又有所谓的“斜体”需求。\LaTeX 也是可以实现这种伪斜体的。虽然可以实现,但排版规范并并不推荐我们使用斜体来强调某个元素。如果想要强调某个元素,可以使用{\heiti 黑体}。

\begin{center}
	{\CJKfontspec[FakeSlant=0.4]{SimSun}\zihao{1}汉字伪斜体}
\end{center}

\begin{latex}{}
{\CJKfontspec[FakeSlant=0.4]{SimSun}\zihao{1}汉字伪斜体}
\end{latex}

西文一般设有加斜,但是这与“斜体”并不是同一回事。加斜是指某种字族的Slant字系;而斜体,是指Italy字族。所以\verb|textit{},\itshape,\it|是斜体(Italic Shape),而\verb|textsl{},\slshape,\sl|是字体的倾斜形状。

\begin{center}
	{\zihao{1} \textit{Italy Shape} ~ \textsl{Slanted Shape}}
\end{center}

\section{长度}

我一般只用如下几个:

\begin{asparadesc}
	\item[em] 当前字号下,大写字母M的宽度,用于水平间距调整
	\item[ex] 当前字号下,小写字母x的高度,用于垂直间距调整
	\item[mm] 毫米,都懂
\end{asparadesc}



\section{空格、换行、分段、分页与对齐}

\LaTeX 将多个空格视为一个,多个换行也会被视为一个。一般习惯使用\verb|~|产生一个空格,使用\verb|\mbox{}|产生一个空白段落(空白行),使用\verb|\par|产生一个带缩进的新段,使用\verb|\\|来强制换行,但下一段的缩进会消失。

段落之间的距离一般这样控制:

\begin{latex}{}
\setlength{\parskip}{0pt plus 1pt}%默认值
\end{latex}

宏包\emph{lettrine}能产生首字下沉效果:

\begin{codeshow}
	\lettrine{吾}{}生也有涯,而知也无涯。
	以有涯随无涯,殆已!
\end{codeshow}

用\verb|\newpage|命令开始新的一页。
用\verb|\clearpage|命令清空浮动体队列5,并开始新的一页。
用\verb|\cleardoublepage|命令清空浮动体队列,并在偶数页上开始新的一页。
注意:以上命令都是基于\verb|\vfill|的。如果要连续新开两页,请在中间加上一个空的箱,如\verb|\newpage\mbox{}\newpage| 。

\LaTeX 默认使用两端对齐来排版,我们可以用\emph{flushleft,flushright,center}这三个环境来构造居左,居右,居中三种版式。特殊情况可以使用\verb|\centering,\raggedleft,\reggedright|来实现居中,居右,居左。







