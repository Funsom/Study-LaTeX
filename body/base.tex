{\let\clearpage\relax \chapter{\LaTeX 基础}}



\section{长度}

我一般只用如下几个:

\begin{asparadesc}
    \item[em] 当前字号下,大写字母M的宽度,用于水平间距调整
    \item[ex] 当前字号下,小写字母x的高度,用于垂直间距调整
    \item[mm] 毫米,都懂
\end{asparadesc}

\section{行距}

\section{页面}

\section{字体}

\begin{asparadesc}
    \item[等宽字体] Typewriter Family 英文的a和i在非等宽字体里面肯定宽度不一样,这样在大段文本里就不好辨认,等宽字体的所以字母宽度一样,笔画的起止还有装饰衬线(所以等宽字体多数属于衬线字体),易读性高
    \item[等线字体] 无字头字脚,笔画圆润,粗细均匀,例如Windows自带的Arial、黑体和幼圆
    \item[衬线字体] serif 在字的笔画开始、结束的地方有额外的装饰,而且笔画的粗细会有所不同,宋体就是一种最标准的serif字体
    \item[无衬线字体] sans serif 在字的笔画开始、结束的地方没有这些额外的装饰,而且笔画的粗细差不多
\end{asparadesc}

等宽字体一般用来书写代码,特别是使用缩进控制语法的\emph{python}语言,更需要等宽字体来书写代码了。

科学书写中文文档的第一步应该是调用\CTeX 宏包,其提供四种命令来调用在中文文档中常用的四种字体。

\begin{latex}
{\songti 爆竹声中一岁除,春风送暖入屠苏。}
{\fangsong 家家乞巧望秋月,穿尽红丝几万条。}
{\heiti 黄沙百战穿金甲,不破楼兰终不还。}
{\kaishu 君不见走马川行雪海边,平沙莽莽黄入天。}
\end{latex}

效果如下:

\begin{center}
    {\zihao{3}\songti 爆竹声中一岁除,春风送暖入屠苏。}\par
    {\zihao{3}\fangsong 家家乞巧望秋月,穿尽红丝几万条。}\par
    {\zihao{3}\heiti 黄沙百战穿金甲,不破楼兰终不还。}\par
    {\zihao{3}\kaishu 君不见走马川行雪海边,平沙莽莽黄入天。}
\end{center}

汉字很少使用粗体和斜体字形,中文文献中的粗体一般用黑体代替,斜体一般用楷书代替。\LaTeX 可以自动做到这一点,当你使用\lstinline|\textbf{文本}|或者\lstinline|\bfseries|这两种粗体命令来强调汉字时,会自动使用黑体汉字做为强调;同样,使用\lstinline|\textit{}|或者\lstinline|\itshape|这两种斜体命令来强调汉字时,会自动使用楷书汉字做为强调。由于xeCJK宏包提供了设置备用字体的功能,所以代码实现比较简单,如下所示:

\begin{latex}
\setCJKmainfont[BoldFont={SimHei},ItalicFont={KaiTi}]{SimSun}
\end{latex}

其中,汉字字体名称可以使用如下命令查找,将列出所有的中文字体的字体族名。

\begin{latex}
fc-list -f "%{family}\n" :lang=zh > zhfont.txt
%常见的中文字体字体族名
Microsoft YaHei,微软雅黑
KaiTi,楷体
SimHei,黑体
LiSu,隶书
YouYuan,幼圆
FangSong,仿宋
SimSun,宋体

STLiti,华文隶书
STSong,华文宋体
STKaiti,华文楷体
STFangsong,华文仿宋
STXingkai,华文行楷
STXihei,华文细黑
STZhongsong,华文中宋
\end{latex}

fontspec和xeCJK也可以使用字体的文件名访问字体。例如Windows下的宋体也可以使用命令:
\begin{latex}
\setCJKmainfont{simsun.ttc}
\end{latex}
来设置。前提是字体已经被安装或者存在与\TeX 索引的目录内,否则需要另行指定路径,这里不再讨论,毕竟学术论文的写作所需字体很少,研究太多并无太大益处。

分全局和局部字体设置。
\subsection{全局字体设置}
中文的文档都要调用\emph{ctex}宏包,该宏包提供一个简单的参数可以设置全部正文的字体。

\begin{latex}
\setmainfont{Times New Roman}    %设置主字体,仅对西文起作用
\setCJKmainfont{SimSun}            %设置主字体,仅对中文起作用
\end{latex}

有时候需要改变\LaTeX 默认的等宽字体,如本文档的等宽字体设置。更改等宽字体之后,将会影响\lstinline|\texttt{}|,\lstinline|\ttfamily|,\lstinline|\tt|这些命令所作用的字体,还会影响默认使用等宽字体(如脚注,抄录环境)的环境。

\begin{latex}
\setmonofont{Source Code Pro}    %英文等宽
\setCJKmonofont{simfang.ttf}    %中文等宽,仿宋
\end{latex}

\subsection{局部字体设置}

\begin{latex}
\newfontfamily\daima{Consolas}    %使用\daima直接调用
\end{latex}

\subsection{在数学环境中使用中文}
默认情况下,数学环境中是不允许输入汉字的。当我们需要输入汉字作为变量的标识时,可以使用\lstinline|\text{要输入的汉字字符}|来完成这项工作。

\begin{codeshow}
$t_{\text{高温}}$
\end{codeshow}

\subsection{汉字“斜体”}
汉字没有加斜体。平常我们看到的加斜汉字,通常是几何变换得到的结果,非常的粗糙,并不严格满足排版要求;而真正的字形是需要精细的设计的。同时,汉字字体里面也很少有加粗体的设计。但是,有时候却又有所谓的“斜体”需求。\LaTeX 也是可以实现这种伪斜体的。虽然可以实现,但排版规范并并不推荐我们使用斜体来强调某个元素。如果想要强调某个元素,可以使用\textbf{黑体}。

\begin{center}
{\CJKfontspec[FakeSlant=0.4]{SimSun}\zihao{1}汉字伪斜体}
\end{center}

\begin{latex}
{\CJKfontspec[FakeSlant=0.4]{SimSun}\zihao{1}汉字伪斜体}
\end{latex}

\section{字符}

在\LaTeX 的文本内容中,大部分字符都可以直接输入,但是\#,\$,\%,\&,\{,\},\_,\^{},\~{},\textless,\textgreater,\textbar,\textbackslash 这几个字符由于有特殊用途不能直接输入。

\begin{codeshow}
\#,\$,\%,\&,\{,\},\_,\^{},\~{},\textless,\textgreater,\textbar,\textbackslash
\end{codeshow}

\bz{英文引号的标准输入法}

英文的单引号并不是两个'符号,双引号也并不是两个''组成的。英文下的引号嵌套需要借助\lstinline{\thinspace}命令分隔。另外,双引号的右半边用"和''的效果是一样的。同样,还可以使用Unicode字符来输入引号,输入方法麻烦,但是更加标准。

\begin{codeshow}
`single quotation marks'\par
``double quotation marks"\par
``\thinspace`Max' is here.''\par
Pumas are ``large, cat-like animals'' which are `found in America'.\par
\textquotedblleft Unicode \textquotedblright \par
\textquoteleft Unicode \textquoteright
\end{codeshow}

\bz{短横}

英文的短横可以产生三种符号:

\begin{compactitem}[\textbullet]
    \item 连字符:输入一个短横,-,效果如daughter-in-law
    \item 数学起止符:输入两个短横,--,效果如page1--2
    \item 英文破折号:输入三个短横:---,效果如Listen---I'm serious
\end{compactitem}

中文破折号,省略号一般直接用中文输入法输入,英文的省略号一般使用\lstinline|\ldots|或者\lstinline|\dots|来输入。

\begin{codeshow}
hello\ldots\par
Thanks\dots
\end{codeshow}

\bz{角度符号,摄氏度符号}
这两个符号需要借助数学模式\lstinline|$...$|来输入:
\begin{codeshow}
$30\,^{\circ}$
$37\,^{\circ}\mathrm{C}$
\end{codeshow}
