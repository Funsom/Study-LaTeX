{\let\clearpage\relax \chapter{\LaTeX 基础}}

\section{\LaTeX 书写环境}

\TeX 环境是\href{http://tug.org/texlive/}{TeX~Live~2016},想要下载速度快可以到清华的镜像站下载。\url{https://mirrors.tuna.tsinghua.edu.cn/CTAN/systems/texlive/Images/}

IDE使用的是\href{http://texstudio.sourceforge.net/}{TeXstudio},整个笔记的目录树如下所示,配有详细的注释。

\dirtree{%
    .1 Study-LaTeX.
    .2 fig\DTcomment{图片文件夹}.
    .3 (name.pdf/png or other suffix)\DTcomment{插入正文的图片}.
    .2 body\DTcomment{章节文件夹}.
    .3 cover.pdf\DTcomment{封面}.
    .3 (chapter name.tex)\DTcomment{章节文件}.
    .2 main.tex\DTcomment{主编译文件}.
    .2 Zousiyu.cls\DTcomment{样式文件}.
    .2 Zousiyu.bib\DTcomment{参考文献数据库}.
    .2 gb7714-2015.bbx\DTcomment{biblatex参考文献样式}.
    .2 gb7714-2015.cbx\DTcomment{biblatex参考文献样式}.
    }

编译使用的是脚本,\lstinline|xelatex.exe main.tex|能直接完成编译,加入\lstinline|--synctex=-1|这个参数可以配置TeXstudio的反向搜索。

\begin{latex}
:: Copyright (c) 2012-2016 Zousiyu

@echo off
:: compile the tex file
xelatex.exe --synctex=-1 main.tex

::pause
biber main

::pause
xelatex.exe --synctex=-1 main.tex

:: clear aux files
call clear
\end{latex}

clear脚本用来清理编译时产生的辅助文件,视情况添加后缀。

\begin{latex}
@echo off
del /q *.aux *.bbl *.bcf *.blg *.listing *.log *.out *.xml *.toc
\end{latex}

\section{长度与间距}
\subsection*{通用长度单位}
首先介绍一下\TeX 中几个通用的长度单位。其中ex、em是\textbf{相对长度单位},其数值大小正比于字体尺寸,当字体尺寸改变,绝对长度会随之改变。其他单位是绝对长度单位。

\begin{table}[!htb]
    \centering
    \caption{\TeX 中常用的长度单位}
    \label{TeX-length}
    \begin{tabular}{ccl}
        \toprule
        单位 & 名称 & 说明\\
        \midrule
        pt & 点 & 欧美传统排版的长度单位,1pt=0.351mm\\
        pc & 派卡 & 相当于四号字大小,1pc=12pt=4.218mm\\
        in & 英寸 & inch英寸,1in=72.27pt=25.4mm\\
        bp & 大点 & big point,1in=72bp)\\
        cm/mm & 都学过 & 1cm=28.453pt,1mm=2.845pt\\
        em & em & 当前字体中M的宽度\\
        ex & ex & 当前字体中x的高度\\
        \bottomrule
    \end{tabular}
\end{table}

\subsection*{专用长度单位}
fil、fill、filll这三个长度单位均表示任意长,伸展能力依次递增。这几种长度单位主要用在长度无法预知或不便计算的情况下,例如将一段文字两侧用空白填满或将版面所剩空间用空白填满。

\subsection*{刚性长度与弹性长度}

\begin{description}
    \item[刚性长度] 不会随排版情况变化而变化的长度,典型的如pt、em等单位。
    \item[弹性长度] 可根据排版长度有一定程度伸缩的长度,如:2mm plus 0.2mm minus 0.3mm,相当于工程标注:$ 2^{+0.2}_{-0.3}mm $。
\end{description}

可伸缩的弹性长度是\LaTeX 的重要排版理念之一。

\subsection*{长度命令}
下面是一些常用的长度命令。

\begin{longtable}{ll}
        \caption{一些常用的产生水平间距的命令}\\
        \toprule
        命令 & 作用\\
        \midrule
        \verb|\quad| & 产生一段宽度为1em 的水平空白\\
        \verb|\qquad| & \verb|\quad|的两倍\\
        \verb|\,| & 大约为\verb|\quad|的3/18\\
        \verb|\hspace{length}| & 产生指定宽度的\textbf{水平空白}\\
        \verb|\hspace*{length}| & 若要在行首产生一定的空白, 则需使用此命令\\
        \verb|\hfill| & 产生撑满整行的空白\\
        \verb|\hphantom{text}| & 幻影命令,产生的空白等于text的宽度\\
        \bottomrule
\end{longtable}

\begin{longtable}{ll}
    \caption{一些常用的产生垂直空白的命令}\\
    \toprule
    命令 & 作用\\
    \midrule
    \verb|\smallskip| & 产生高度为3pt plus 1pt minus 1pt的垂直空白\\
    \verb|\medskip| & \verb|\smallskip|的两倍\\
    \verb|\bigskip| & \verb|\smallskip|的四倍\\
    \verb|\vspace{length}| & 产生指定高度的垂直空白\\
    \verb|\vspace*{length}| & 在页面顶部产生垂直空白\\
    \verb|\vfill| & 插入指定高度的垂直空白\\
    \verb|\vphantom{text}| & 幻影命令,产生的空白等于text的高度,和text的长度无关的\\
    \bottomrule
\end{longtable}

\subsection{正确处理单词间距}
英文排版时\TeX{}通常默认句号\code{.}表示一句话的结束,\textbf{因此\TeX{}在处理句号时会留出稍宽一点的水平间距}。但是有些情况下,句号并不代表句子的结尾,比如「i.e. a word」和「e.g. a word」。按照\TeX{}默认规则,排版出的宽度会比正常句中单词之间的间隔稍大一些,因此我们需要使用使用\code{\}(即一个反斜杠+空格)来消除这个过大的间距:\code{i.e.\ a word}以及\code{e.g.\ a word}。仔细观察下面例子的排版效果。

\begin{codeshow}
i.e. a word\par
i.e.\ a word
\end{codeshow}

句号跟在一个大写字母的后面,\textbf{此时\TeX{}会认为这个句号表示人名缩写的间隔符},因此仍然按照正常间距来排版,比如 「A. Einstein」。然而这个看似贴心的规则在一些情况下会适得其反,比如一句话明明以缩略语结尾,\TeX{}反而认为这并不是一句话的结尾:「... in NBA. He...」。此时,排版出的「He」之前的空格会小于正常的句间间距。这种情况下,需要使用 \code{\@.}(反斜杠+@+句号+空格),来强制告诉\TeX{}这里的的确确是一个句子的结尾。

\begin{codeshow}
... played in NBA. He was ...\par
... played in NBA\@. He was ...
\end{codeshow}

以上规则除句号外,同样适用于感叹号和问号等其他符号。

\subsection{数学符号中的间距}

数学公式中,积分符号$ dx $\footnote{积分符号是直立还是斜体尚有争论}前应该加入一个间距\code{!,},同时在公式结尾的标定符号与公式之间也应该插入一个间距\code{!,}。此外,积分符号$ \int $与被积分项之间的间距在默认情况下过大,完美的排版需要利用\code{\!}来缩小这个间距。如下$ \int $与$ f(x) $,$ f(x) $与$ dx $,$ \alpha $与$ . $之间的间距都值得注意。

\begin{codeshow}
\[ \int_a^bf(x)dx = \alpha. \]\par
\[ \int_a^b\!f(x)\,dx = \alpha\,. \]
\end{codeshow}

\section{行、段落、页面}

\subsection*{避免数字出现在行首}
使用\lstinline|~|来代替空格可以避免交叉引用或者输入人名时尴尬地被打破成两行,例如\lstinline|...如图~\ref{Fig1}所示...|,或者\lstinline|...A.~Einstein said...|。

\subsection*{中英混排时空格的使用}
中英文混排时,\XeLaTeX{}能在中文与英文(或数字)之间,没有必要手动敲入一个空格,编译时会自动为中文与英文(或数字)之间添加合适的间距。但是有一个情况比较特殊,就是在交叉引用时,这个空格是需要手动敲入的,否则这个间距会消失。

\begin{description}[labelwidth=4em]
    \item[换行] \LaTeX{}会自动换行, 若需强制换行, 可使用\lstinline|\\|或\lstinline|\newline|。\lstinline|\\|后面可以带长度,以增加当前行与新行之间的距离,参数可正可负,如:\lstinline|\\[3mm]|,\lstinline|\\[-5pt]|。
    \item[分段] 两个连续回车(即一个空行)或\lstinline|\par|。
    \item[分页] \LaTeX{}会自动分页若需强制分页, 可用命令\lstinline|\newpage|或\lstinline|\clearpage|。
\end{description}

\section{页面}

\section{字体}

\begin{description}
    \item[等宽字体] Typewriter Family 英文的a和i在非等宽字体里面肯定宽度不一样,这样在大段文本里就不好辨认,等宽字体的所以字母宽度一样,笔画的起止还有装饰衬线(所以等宽字体多数属于衬线字体),易读性高
    \item[等线字体] 无字头字脚,笔画圆润,粗细均匀,例如Windows自带的Arial、黑体和幼圆
    \item[衬线字体] serif 在字的笔画开始、结束的地方有额外的装饰,而且笔画的粗细会有所不同,宋体就是一种最标准的serif字体
    \item[无衬线字体] sans serif 在字的笔画开始、结束的地方没有这些额外的装饰,而且笔画的粗细差不多
\end{description}

等宽字体一般用来书写代码,特别是使用缩进控制语法的\emph{python}语言,更需要等宽字体来书写代码了。

科学书写中文文档的第一步应该是调用\CTeX 宏包,其提供四种命令来调用在中文文档中常用的四种字体。

\begin{latex}
{\songti 爆竹声中一岁除,春风送暖入屠苏。}
{\fangsong 家家乞巧望秋月,穿尽红丝几万条。}
{\heiti 黄沙百战穿金甲,不破楼兰终不还。}
{\kaishu 君不见走马川行雪海边,平沙莽莽黄入天。}
\end{latex}

效果如下:

\begin{center}
    {\zihao{3}\songti 爆竹声中一岁除,春风送暖入屠苏。}\par
    {\zihao{3}\fangsong 家家乞巧望秋月,穿尽红丝几万条。}\par
    {\zihao{3}\heiti 黄沙百战穿金甲,不破楼兰终不还。}\par
    {\zihao{3}\kaishu 君不见走马川行雪海边,平沙莽莽黄入天。}
\end{center}

汉字很少使用粗体和斜体字形,中文文献中的粗体一般用黑体代替,斜体一般用楷书代替。\LaTeX 可以自动做到这一点,当你使用\lstinline|\textbf{文本}|或者\lstinline|\bfseries|这两种粗体命令来强调汉字时,会自动使用黑体汉字做为强调;同样,使用\lstinline|\textit{}|或者\lstinline|\itshape|这两种斜体命令来强调汉字时,会自动使用楷书汉字做为强调。由于xeCJK宏包提供了设置备用字体的功能,所以代码实现比较简单,如下所示:

\begin{latex}
\setCJKmainfont[BoldFont={SimHei},ItalicFont={KaiTi}]{SimSun}
\end{latex}

其中,汉字字体名称可以使用如下命令查找,将列出所有的中文字体的字体族名。

\begin{latex}
fc-list -f "%{family}\n" :lang=zh > zhfont.txt
%常见的中文字体字体族名
Microsoft YaHei,微软雅黑
KaiTi,楷体
SimHei,黑体
LiSu,隶书
YouYuan,幼圆
FangSong,仿宋
SimSun,宋体

STLiti,华文隶书
STSong,华文宋体
STKaiti,华文楷体
STFangsong,华文仿宋
STXingkai,华文行楷
STXihei,华文细黑
STZhongsong,华文中宋
\end{latex}

fontspec和xeCJK也可以使用字体的文件名访问字体。例如Windows下的宋体也可以使用命令:
\begin{latex}
\setCJKmainfont{simsun.ttc}
\end{latex}
来设置。前提是字体已经被安装或者存在与\TeX 索引的目录内,否则需要另行指定路径,这里不再讨论,毕竟学术论文的写作所需字体很少,研究太多并无太大益处。

分全局和局部字体设置。
\subsection{全局字体设置}
中文的文档都要调用\emph{ctex}宏包,该宏包提供一个简单的参数可以设置全部正文的字体。

\begin{latex}
\setmainfont{Times New Roman}    %设置主字体,仅对西文起作用
\setCJKmainfont{SimSun}            %设置主字体,仅对中文起作用
\end{latex}

有时候需要改变\LaTeX 默认的等宽字体,如本文档的等宽字体设置。更改等宽字体之后,将会影响\lstinline|\texttt{}|,\lstinline|\ttfamily|,\lstinline|\tt|这些命令所作用的字体,还会影响默认使用等宽字体(如脚注,抄录环境)的环境。

\begin{latex}
\setmonofont{Source Code Pro}    %英文等宽
\setCJKmonofont{simfang.ttf}    %中文等宽,仿宋
\end{latex}

\subsection{局部字体设置}

\begin{latex}
\newfontfamily\daima{Consolas}    %使用\daima直接调用
\end{latex}

\subsection{在数学环境中使用中文}
默认情况下,数学环境中是不允许输入汉字的。当我们需要输入汉字作为变量的标识时,可以使用\lstinline|\text{要输入的汉字字符}|来完成这项工作。

\begin{codeshow}
$t_{\text{高温}}$
\end{codeshow}

\subsection{汉字“斜体”}
汉字没有加斜体。平常我们看到的加斜汉字,通常是几何变换得到的结果,非常的粗糙,并不严格满足排版要求;而真正的字形是需要精细的设计的。同时,汉字字体里面也很少有加粗体的设计。但是,有时候却又有所谓的“斜体”需求。\LaTeX 也是可以实现这种伪斜体的。虽然可以实现,但排版规范并并不推荐我们使用斜体来强调某个元素。如果想要强调某个元素,可以使用\textbf{黑体}。

\begin{center}
{\CJKfontspec[FakeSlant=0.4]{SimSun}\zihao{1}汉字伪斜体}
\end{center}

\begin{latex}
{\CJKfontspec[FakeSlant=0.4]{SimSun}\zihao{1}汉字伪斜体}
\end{latex}

\section{字符}

在\LaTeX 的文本内容中,大部分字符都可以直接输入,但是\#,\$,\%,\&,\{,\},\_,\^{},\~{},\textless,\textgreater,\textbar,\textbackslash 这几个字符由于有特殊用途不能直接输入。

\begin{codeshow}
\#,\$,\%,\&,\{,\},\_,\^{},\~{},\textless,\textgreater,\textbar,\textbackslash
\end{codeshow}

\bz{英文引号的标准输入法}

英文的单引号并不是两个'符号,双引号也并不是两个''组成的。英文下的引号嵌套需要借助\lstinline{\thinspace}命令分隔。另外,双引号的右半边用"和''的效果是一样的。同样,还可以使用Unicode字符来输入引号,输入方法麻烦,但是更加标准。

\begin{codeshow}
``\thinspace`Max' is here.''\par
Pumas are ``large, cat-like animals'' which are `found in America'.\par
\textquotedblleft Unicode \textquotedblright \par
\textquoteleft Unicode \textquoteright
\end{codeshow}

用一个例子解释一下为什么英文的引号需要这样输入。\textbf{能看出'打出的都是右引号!}

\bz{英文引号的错误用法}

\begin{codeshow}
'wrong'\\
`right'
\end{codeshow}

\subsection*{短横}

英文的短横可以产生三种符号:

\begin{description}
    \item[连字符] \textbf{通常用来连接复合词},输入一个短横,\code{-},效果如daughter-in-law
    \item[数学起止符] \textbf{通常用来表示范围},输入两个短横,\code{--},效果如page 1--2,如果真的希望连续输入两个连字符,使用\code{{-}{-}}
    \item[英文破折号] 是一个正规的标点符号,用来表示转折或者承上启下。破折号与其前后的单词之间不应该存在空格,输入三个短横:\code{---},效果如Listen---I'm serious
\end{description}

\note{排版中的减号应该比连字符要长,因此用来表示减号或者负号时,请严格使用数学模式而不要使用文字模式。}

以上符号区别如下,注意前面讲过的数学符号中的间距这个小细节:

\begin{codeshow}
daughter-in-law\par
page 1--2\par
Listen---I'm serious\par
The temperature is $ -5\,^{\circ}\mathrm{C} $
\end{codeshow}

\subsection*{省略号}

中文破折号,省略号一般直接用中文输入法输入,英文的省略号一般使用\lstinline|\ldots|或者\lstinline|\dots|来输入。

\begin{codeshow}
hello\ldots\par
Thanks\dots
\end{codeshow}

\subsection*{摄氏度}

\bz{角度符号,摄氏度符号}
这两个符号需要借助数学模式\lstinline|$...$|来输入:
\begin{codeshow}
$30\,^{\circ}$\\
$37\,^{\circ}\mathrm{C}$
\end{codeshow}
