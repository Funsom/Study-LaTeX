{\let\clearpage\relax \chapter{宏包}}

\LaTeX 的强大之处在于有各种各样的宏包进行扩展,这些宏包能帮助\LaTeX 完成多样的排版任务。

例如,在文本中排版化学式,有两个宏包比较流行,\emph{mhchem}和\emph{chemfig}。前者功能简单,用法也简单,书写无机化学式首选;后者功能众多,用法也较为繁复,主要用来书写复杂的有机化学式。

\section{mhchem}

使用下面的语句调用\emph{mhchem}宏包,添加\verb|version=4|参数是为了使用宏包的一些新特性,编写的文档如果要照顾老旧的\LaTeX 版本用户,可以酌情降低版本。

\begin{latex}
\usepackage[version=4]{mhchem}
\end{latex}

\bz{化学方程式}

\begin{codeshow}
\ce{CO2 + C -> 2 CO}\par
\ce{Hg^2+ ->[I-] HgI2
    ->[I-] [Hg^{II}I4]^2-}
\end{codeshow}

\bz{化学分子式}

分子式可用在文本模式、数学模式和标题里面。

\begin{codeshow}
\ce{H2O}\par
\ce{Sb2O3}
 \end{codeshow}

\bz{离子}

\begin{codeshow}
\ce{H+}\par
\ce{CrO4^2-}\par
\ce{[AgCl2]-}\par
\ce{Y^99+}\par
\ce{Y^{99+}}
\end{codeshow}

\bz{氧化价态}

\begin{codeshow}
\ce{Fe^{II}Fe^{III}2O4}
\end{codeshow}

\bz{化学计量数}

\begin{codeshow}
\ce{2H2O}\par
\ce{2 H2O}\par
\ce{0.5H2O}\par
\ce{1/2H2O}\par
\ce{(1/2)H2O}\par
\ce{$n$H2O}
\end{codeshow}

\bz{同位素}

\begin{codeshow}
\ce{^{227}_{90}Th+}\par
\ce{^227_90Th+}\par
\ce{^{0}_{-1}n^{-}}\par
\ce{^0_-1n-}
\end{codeshow}

\bz{括弧,方括号,花括号}

\verb|(),[]|可以正常表示,但需要使用\verb|\{ \}|来表示花括号。大型的花括号只能在数学环境里面使用。

\begin{codeshow}
\ce{(NH4)2S}\par
\ce{[\{(X2)3\}2]^3+}
\end{codeshow}

\bz{大花括号}
\begin{latex}
\[\ce{CH4 + 2 $\left( \ce{O2 + 79/21 N2} \right)$}\]
\end{latex}

\[\ce{CH4 + 2 $\left( \ce{O2 + 79/21 N2} \right)$}\]

\bz{聚合态}

\begin{codeshow}
\ce{H2(aq)}\par
\ce{CO3^2-{}_{(aq)}}\par
\ce{NaOH(aq,$\infty$)}
\end{codeshow}

\bz{未成对电子,自由基}

\begin{codeshow}
\ce{OCO^{.-}}\par
\ce{NO^{(2.)-}}
\end{codeshow}

\bz{变量}
排版约定,变量使用斜体排版,而其他元素(如化学式)则使用直立字体排版。

\begin{codeshow}
$\ce{NO_x}$\par
$\ce{Fe^n+}$
\end{codeshow}

\begin{latex}
$\ce{x Na(NH4)HPO4 ->[\Delta] (NaPO3)_x + x NH3 ^ + x H2O}$
\end{latex}

\begin{center}
    $\ce{x Na(NH4)HPO4 ->[\Delta] (NaPO3)_x + x NH3 ^ + x H2O}$
\end{center}

\bz{希腊字符}
\begin{codeshow}
\ce{\mu-Cl}\par
\ce{[Pt(\eta^2-C2H4)Cl3]-}
\end{codeshow}

\bz{加成化合物}
\begin{codeshow}
\ce{KCr(SO4)2*12H2O}\par
\ce{KCr(SO4)2.12H2O}\par
\ce{KCr(SO4)2 * 12 H2O}
\end{codeshow}

\bz{化学键}
\begin{codeshow}
\ce{C6H5-CHO}\par
\ce{A-B=C#D}\par
\sffamily\bfseries\ce{A-B=C#D}
\end{codeshow}

\begin{codeshow}
\ce{A\bond{-}B\bond{=}C\bond{#}D}\par
\ce{A\bond{1}B\bond{2}C\bond{3}D}\par
\ce{A\bond{~}B\bond{~-}C}\par
\ce{A\bond{~--}B\bond{~=}C\bond{-~-}D}\par
\ce{A\bond{...}B\bond{....}C}\par
\ce{A\bond{->}B\bond{<-}C}
\end{codeshow}

\bz{反应箭头}
\begin{codeshow}
\ce{A -> B}\par
\ce{A <- B}\par
\ce{A <-> B}\par
\ce{A <--> B}\par
\ce{A <=>> B}\par
\ce{A <<=> B}
\end{codeshow}

\bz{带参数的反应箭头}
\begin{codeshow}
\ce{A ->[H2O] B}\par
\ce{A ->[{text above}][{text below}] B}\par
\ce{A ->[$x$][$x_i$] B}\par
\ce{A ->[${x}$] B}
\end{codeshow}

\bz{化学方程式计算符}
\begin{codeshow}
\ce{A + B}\par
\ce{A - B}\par
\ce{A = B}\par
\ce{A \pm B}
\end{codeshow}

\bz{沉淀和气体}
\begin{codeshow}
\ce{SO4^2- + Ba^2+ -> BaSO4 v}\par
\ce{A v B (v) -> B ^ B (^)}
\end{codeshow}

\bz{极好的示例}

\begin{latex}
\ce{Zn^2+
    <=>[+ 2OH-][+ 2H+]
    $\underset{\text{amphoteres Hydroxid}}{\ce{Zn(OH)2 v}}$
    <=>[+ 2OH-][+ 2H+]
    $\underset{\text{Hydroxozikat}}{\ce{[Zn(OH)4]^2-}}$
}

$K = \frac{[\ce{Hg^2+}][\ce{Hg}]}{[\ce{Hg2^2+}]}$

$K = \ce{\frac{[Hg^2+][Hg]}{[Hg2^2+]}}$

\ce{Hg^2+ ->[I-]
    $\underset{\mathrm{red}}{\ce{HgI2}}$
    ->[I-]
    $\underset{\mathrm{red}}{\ce{[Hg^{II}I4]^2-}}$
}
\end{latex}

\begin{center}
\ce{Zn^2+
    <=>[+ 2OH-][+ 2H+]
    $\underset{\text{amphoteres Hydroxid}}{\ce{Zn(OH)2 v}}$
    <=>[+ 2OH-][+ 2H+]
    $\underset{\text{Hydroxozikat}}{\ce{[Zn(OH)4]^2-}}$
}

$K = \frac{[\ce{Hg^2+}][\ce{Hg}]}{[\ce{Hg2^2+}]}$

$K = \ce{\frac{[Hg^2+][Hg]}{[Hg2^2+]}}$

\ce{Hg^2+ ->[I-]
    $\underset{\mathrm{red}}{\ce{HgI2}}$
    ->[I-]
    $\underset{\mathrm{red}}{\ce{[Hg^{II}I4]^2-}}$
}
\end{center}

\section{Chemfig}

\section{\CTeX zhnumber}

以中文格式输出数字。这里的数字可以是整数、小数和分数。

\begin{codeshow}
\zhnumber{2012020120}\\
\zhnumber{2 012 020 120}\\
\zhnumber{2,012,020,120}\\
\zhnumber{2012.020120}\\
\zhnumber{2012.}\\
\zhnumber{.2012}\\
\zhnumber{20120/20120}\\
\zhnumber{/2012}\\
\zhnumber{2012/}\\
\zhnumber{201;2020/120}
\end{codeshow}

\bz{带*才能映射0为零}

将阿拉伯数字转换为中文字符串。
\begin{codeshow}
\zhdigits{2012020120}\\
\zhdigits*{2012020120}
\end{codeshow}

将\LaTeX 计数器数值转换为中文。

\begin{latex}
\zhnum{section}
\end{latex}

输出当天的星期。

\begin{codeshow}
\zhweekday{2012/5/20}
\end{codeshow}

\bz{带*的可以输出星期}
以中文格式输出日期。

\begin{codeshow}
\zhdate{2012/5/21}\\
\zhdate*{1995/01/26}
\end{codeshow}

以中文输出当天日期。

\begin{codeshow}
\zhtoday
\end{codeshow}

以中文输出时间。

\begin{codeshow}
\zhtime{23:56}
\end{codeshow}

输出当前时间。

\begin{codeshow}
\zhcurrtime
\end{codeshow}

输出天干计数,数字范围是1--10。

\begin{codeshow}
\zhtiangan{1} \zhtiangan{2}
\zhtiangan{3} \zhtiangan{4}
\zhtiangan{5} \zhtiangan{10}
\end{codeshow}

输出地支计数,数字范围1--12。

\begin{codeshow}
\zhdizhi{1} \zhdizhi{2} \zhdizhi{3}
\zhdizhi{4} \zhdizhi{5} \zhdizhi{12}
\end{codeshow}

输出干支计数,数字范围1--60。

\begin{codeshow}
\zhganzhi{1} \zhganzhi{2}
\zhganzhi{3}\\ \zhganzhi{4}
\zhganzhi{5} \zhganzhi{60}
\end{codeshow}

输出公元纪年对应的干支纪年,公元前用负数。

\begin{codeshow}
\zhganzhinian{1898}
\zhganzhinian{-246} \\
\zhganzhinian{-2697}
\zhganzhinian{\year}
\end{codeshow}

\emph{zhnumsetup}的样式控制选项。

\begin{asparadesc}
    \item [Simplified] 以简体中文输出数字(对Big5 编码无效)
    \item [Traditional] 以繁体中文输出数字(对Big5 编码无效)
    \item [Normal] 以小写形式输出中文数字
    \item [Financial] 以大写形式输出中文数字
    \item [Ancient] 以廿输出20,以卅输出30,以卌输出40,以皕输出200
\end{asparadesc}

\begin{codeshow}
\zhnumsetup{
    style={Traditional,Normal}}
\zhnumber{62012.3}\\
\zhnumsetup{style=Ancient}
\zhnumber{21}
\end{codeshow}

\begin{codeshow}
\zhnumsetup{
    style={Traditional,Financial}}
\zhnumber{62012.3}\\
\zhnumsetup{style=Ancient}
\zhnumber{21}
\end{codeshow}


\section{siunitx}
这是一个用于书写标准国际单位制的宏包。

\section{pgfplots}
这是一个用于2D/3D图像绘制的宏包。
















