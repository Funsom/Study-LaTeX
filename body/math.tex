{\let\clearpage\relax \chapter{数学排版}}

终于到了\LaTeX 最擅长的部分,数学排版。

\section{数学模式}
分行内公式和行间公式。

行内公式,即:\verb|$\sum_{i=1}^{n}a_i$|,得到:$\sum_{i=1}^{n}a_i$.

行间公式,即:\verb|\[\sum_{i=1}^n{a_i}\]|,得到:
\[\sum_{i=1}^{n}a_i\]

\section{数学宏包}

\section{数学符号}
\subsection{上标与下标}
上下标一般写在数学符号的右上、右下方,如果需要将它们写在正下、正上方,可以使用\emph{\textbackslash limits}。
\begin{codeshow}
$A_{ij}=2^{i+j}$\par
$\sum_{i=1}^{n}$\par
$\sum\limits_{i=1}^{n}$\par
\end{codeshow}

如果是行间公式,上下标默认就在正下、正上方。另外,使用\emph{\textbackslash substack}命令可以加入多行的上下标,举个例子。
\begin{codeshow}
\[\sum_{min}^{max}\]
\[\sum_{\substack{i=1\\j=1}}^{n}\]
\end{codeshow}

\subsection{画线补充}
想划线,就拿\emph{\textbackslash overline}和\emph{\textbackslash underline}命令就可以了,划线的部分最好以花括号括起来。想画箭头则将line替换为arrow。想打双向箭头或其他,那么把left/right改成leftright\footnote{连写,先left后right}。举个例子。
\begin{codeshow}
$ \overleftarrow{abc} $\par
$ \underline{xy} $\par
$ a \leftrightarrow b $\par
$ \overleftrightarrow{abc} $
\end{codeshow}

如果想在数学环境里面写中文\footnote{并不推荐这样做,只是为了符合国情才有教材在数学公式里面排版中文},那么记住两件事,一是在开头引用\CTeX 宏包,二是在引用中文的之前使用\emph{\textbackslash text}命令,举一个例子。
\begin{codeshow}
$\overbrace{(a_0,a_1,\dots,a_n)}
^{\text{共 $n+1$ 项}}$
%\dots命令能在基线上产生三个点
\end{codeshow}

\subsection{分式}
使用命令\emph{\textbackslash frac}写出正常的分式而不是a/b这种的,命令之后有两个参数,如果分子分母均只有一个字符,则可以不加花括号。举例如下。
\begin{codeshow}
%行内公式形式
$\frac12 \quad \frac2n
\quad \frac{2}{2+n}$
%行间公式形式
\[ \frac12 \quad \frac2n
\quad \frac{2}{2+n} \]
\end{codeshow}

如果你想玩点花样,随意使用行内公式和行间公式,那么这里的\emph{\textbackslash frac}可以分支为\emph{\textbackslash dfrac}和\emph{\textbackslash tfrac},t即text(行内,文本模式),d即display(行间,显示模式)。我们可以用这两个命令调节嵌套分式的大小,举个例子。
\begin{codeshow}
\[ \frac{1}{\tfrac1a+\tfrac1b+c} \]
\[ \frac{1}{\dfrac1a+\dfrac1b+c} \]
\end{codeshow}

\subsection{根式}
开方的次数\footnote{非数学专业,不知道用次数表达是否合理,欢迎指正}用方括号[]括起来。注意,根式的开方次数如果过大,写在左边就很影响美观,这个时候一般都改为指数形式。
\begin{codeshow}
\[
\sqrt{x^2+1}\quad \sqrt[3]{x^4+1}
\]
\end{codeshow}

\subsection{嵌套}
所有的公式都可以做到嵌套,这样子就可以形成相对比较复杂的公式。
\begin{codeshow}
\[
\frac{-b\pm \sqrt{b^{2}-4ac}}{2a}\\
\lim\limits_{x\to 0}\frac
{x\cdot \frac{\cos x -1}{\cos x}}{x^{3}}
\]
\end{codeshow}

除了在分式中会经常用到嵌套以外,矩阵里这种情况也很常见,比如分块矩阵,举个例子。当然,我们也可以把零弄大一点,我们只需要将0修改为\emph{\textbackslash text{large{0}}}就好
\begin{codeshow}
\[
A=\begin{pmatrix}
\begin{matrix}
1 & 0 \\
0 & 1
\end{matrix} & 0 \\
\text{\large{0}} & \begin{matrix}
1 & 0 \\
0 & 1
\end{matrix}
\end{pmatrix}
\]
\end{codeshow}

\subsection{定界符}
嵌套多了式子会变得非常复杂,也就会变得越来越大!可是这个时候如果你使用括号你会发现,它的大小并没有什么变化,这就显得非常的low,影响美观,因此我们会在括号外加一个left或者是right进行大小的控制。举例如下。
\begin{codeshow}
\[
\lim\limits_{x\to 0}\left(\frac
{a^{x}+b^{x}+c^{x}}{3}\right)
^{\tfrac{1}{x}}
\]
\end{codeshow}

学了定界符之后,就可以完全实现矩阵的部分形态了,比方说排版一个增广矩阵。

\begin{codeshow}
\[
\left(
\begin{tabular}{ccc|c}
1 & 1 & 1 & 1 \\
1 & 1 & 1 & 1 \\
1 & 1 & 1 & 1 
\end{tabular}
\right)
\]
\end{codeshow}

\subsection{数学字体}
标准的\LaTeX 提供的数学字体有以下几种。简单的文档中,这些字体已经够用了,如果要使用更高级的字体,可查阅\CTeX 宏包说明。
\begin{codeshow}
\[
\mathit{ABCDE}\]
\[
\mathrm{ABCDE}\\\]
\[
\mathbf{ABCDE}\\\]
\[
\mathsf{ABCDE}\\\]
\[
\mathtt{ABCDE}\\\]
\end{codeshow}

\subsection{希腊字母}
有时间排个表在这,不着急。

\subsection{函数符号}
调用\emph{amsmath}宏包后,大多数函数能够使用反斜杠加名称直接打出,例如:

\begin{codeshow}
$ \sin \quad \ln \quad \arccos $
\end{codeshow}

像$ \arcsec \quad \arccot \quad \arccsc $这三个函数,\emph{amsmath}宏包就没有定义,这就需要我们自己定义这样一个新的函数命令。
\begin{latex}
%定义一些amsmath没有定义的函数
\DeclareMathOperator{\arcsec}{arcsec}
\DeclareMathOperator{\arccot}{arccot}
\DeclareMathOperator{\arccsc}{arccsc}
\end{latex}

\section{公式环境}

\subsection{align}
align环境的对齐功能。
\begin{latex}
\begin{align*}
&\lim\limits_{x\to 1}\left(\frac{1}{1-x}-\frac{3}{1-x^3}\right)\\
= &\lim\limits_{x\to 1}\left(\frac{x^2+x-2}{1-x^3}\right)  \\
=& \lim\limits_{x\to 1}\frac{(x+2)(x-1)}{(1-x)(x^2+x+1)}\\
=& \lim\limits_{x\to 1}\frac{-(x+2)}{x^2+x+1}\\
=& -1
\end{align*}
\end{latex}

\begin{align*}
&\lim\limits_{x\to 1}\left(\frac{1}{1-x}-\frac{3}{1-x^3}\right)\\
= &\lim\limits_{x\to 1}\left(\frac{x^2+x-2}{1-x^3}\right)  \\
=& \lim\limits_{x\to 1}\frac{(x+2)(x-1)}{(1-x)(x^2+x+1)}\\
=& \lim\limits_{x\to 1}\frac{-(x+2)}{x^2+x+1}\\
=& -1
\end{align*}

\subsection{多行公式}

\begin{latex}
\begin{multline}
\frac{\rho}{\epsilon_p} \left(\frac{\partial u}{\partial t}+(u \bullet \nabla)\frac{u}{\epsilon_p}\right)=\\
\nabla \bullet \left[ -pl+\frac{\mu}{\epsilon_p}\left( \nabla u+(\nabla u)^T\right)-\frac{2\mu}{3\epsilon_p}(\nabla \bullet u)l\right]-\left( \mu \kappa^{-1} + \beta_{F}u+ \frac{Q_br}{\epsilon^{2}_{p}} \right)u+F
\end{multline}
\end{latex}

\begin{multline}
\frac{\rho}{\epsilon_p} \left(\frac{\partial u}{\partial t}+(u \bullet \nabla)\frac{u}{\epsilon_p}\right)=\\
\nabla \bullet \left[ -pl+\frac{\mu}{\epsilon_p}\left( \nabla u+(\nabla u)^T\right)-\frac{2\mu}{3\epsilon_p}(\nabla \bullet u)l\right]-\left( \mu \kappa^{-1} + \beta_{F}u+ \frac{Q_br}{\epsilon^{2}_{p}} \right)u+F
\end{multline}

\section{矩阵环境}
矩阵的环境和表格有点相似,所以用法也和列表几乎相同,举个最简单的矩阵例子。
\begin{codeshow}
\[
A=\begin{matrix}
a_{11} & a_{12} & a_{13}\\
0 & 0 & 0\\
0 & 0 & 0
\end{matrix}
\]
\end{codeshow}

那我要写带括号的呢?没关系,不同的矩阵环境会形成不同的括号。这里的matrix就不形成括号,pmatrix形成小括号,bmatrix形成中括号,vmatrix形成竖线(行列式形式),Bmatrix形成大括号,Vmatrix形成双竖线。
\begin{codeshow}
\begin{gather*}
    %居中的公式组环境,不编号
    \begin{pmatrix}1 & 2\\
    3 & 4\end{pmatrix}
    \begin{bmatrix}1 & 2\\
    3 & 4\end{bmatrix}
    \begin{vmatrix}1 & 2\\
    3 & 4\end{vmatrix}\\
    \begin{Bmatrix}1 & 2\\
    3 & 4\end{Bmatrix}
    \begin{Vmatrix}1 & 2\\
    3 & 4\end{Vmatrix}
\end{gather*}
\end{codeshow}

矩阵的元素有时候会很多,需要使用省略号去忽略,而省略号在tex中有专门的命令,列举如下。
\begin{codeshow}
\[
\ldots \cdots \vdots \ddots \dotsc
\]
\end{codeshow}




\section{定理环境}