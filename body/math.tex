\newpage
\chapter{数学排版}
\thispagestyle{chapterpage}

终于到了\LaTeX 最擅长的部分,数学排版。

\section{数学模式}
分行内公式和行间公式。

行内公式,即:\lstinline|$\sum_{i=1}^{n}a_i$|,得到:$\sum_{i=1}^{n}a_i$.

行间公式,即:\lstinline|\[\sum_{i=1}^n{a_i}\]|,得到:
\[\sum_{i=1}^{n}a_i\]

\section{数学宏包}

\section{数学符号}
\subsection{上标与下标}
上下标一般写在数学符号的右上、右下方,如果需要将它们写在正下、正上方,可以使用\lstinline|\limits|。
\begin{codeshow}
$A_{ij}=2^{i+j}$\par
$\sum_{i=1}^{n}$\par
$\sum\limits_{i=1}^{n}$\par
\end{codeshow}

如果是行间公式,上下标默认就在正下、正上方。另外,使用\lstinline|\substack|命令可以加入多行的上下标,举个例子。
\begin{codeshow}
\[\sum_{min}^{max}\]
\[\sum_{\substack{i=1\\j=1}}^{n}\]
\end{codeshow}

\subsection{画线补充}
想划线,就拿\lstinline|\overline|和\lstinline|\underline|命令就可以了,划线的部分最好以花括号括起来。想画箭头则将line替换为arrow。想打双向箭头或其他,那么把left/right改成leftright\footnote{连写,先left后right}。举个例子。
\begin{codeshow}
$ \overleftarrow{abc} $\par
$ \underline{xy} $\par
$ a \leftrightarrow b $\par
$ \overleftrightarrow{abc} $
\end{codeshow}

如果想在数学环境里面写中文\footnote{并不推荐这样做,只是为了符合国情才有教材在数学公式里面排版中文},那么记住两件事,一是在开头引用\CTeX 宏包,二是在引用中文的之前使用\lstinline|\text|命令,举一个例子。
\begin{codeshow}
$\overbrace{(a_0,a_1,\dots,a_n)}
^{\text{共 $n+1$ 项}}$
%\dots命令能在基线上产生三个点
\end{codeshow}

\subsection{分式}
使用命令\lstinline|\frac|写出正常的分式而不是a/b这种的,命令之后有两个参数,如果分子分母均只有一个字符,则可以不加花括号。举例如下。
\begin{codeshow}
%行内公式形式
$\frac12 \quad \frac2n
\quad \frac{2}{2+n}$
%行间公式形式
\[ \frac12 \quad \frac2n
\quad \frac{2}{2+n} \]
\end{codeshow}

如果你想玩点花样,随意使用行内公式和行间公式,那么这里的\lstinline|\frac|可以分支为\lstinline|\dfrac|和\lstinline|\tfrac|,t即text(行内,文本模式),d即display(行间,显示模式)。我们可以用这两个命令调节嵌套分式的大小,举个例子。
\begin{codeshow}
\[ \frac{1}{\tfrac1a+\tfrac1b+c} \]
\[ \frac{1}{\dfrac1a+\dfrac1b+c} \]
\end{codeshow}

\subsection{斜线分式和斜线除号}

对于一些需要用到斜除号的地方,如果斜除号两边的字符比较高,用常规的\code{/}会导致式子很不协调,这个时候可以使用\code{\middle/}来使得斜除号的高度与两侧字符高度相匹配。如下所示:

\begin{codeshow}
$x=a^\frac{1}{2}/b$\\
$x=\left.a^\frac{1}{2}\middle/b\right.$
\end{codeshow}

此外,有时候还需要用到行内斜线分式.通常,我们输入斜线分数都是键入X/Y,但是这个真心有点难看。我们可以用专业的xfrac宏包来处理这些斜线分式,它提供一个命令:

\begin{latex}
\sfrac{}{}
\end{latex}

\note{这个命令可以在数学环境外使用,即在文本模式中直接使用}

\begin{codeshow}
\sfrac{1}{2}\\
$ \sfrac{7}{20} $\\
$\sfrac{1}{4}$ cups of sugar
\end{codeshow}

\subsection{根式}
开方的次数\footnote{非数学专业,不知道用次数表达是否合理,欢迎指正}用方括号[]括起来。注意,根式的开方次数如果过大,写在左边就很影响美观,这个时候一般都改为指数形式。
\begin{codeshow}
\[
\sqrt{x^2+1}\quad \sqrt[3]{x^4+1}
\]
\end{codeshow}

\subsection{嵌套}
所有的公式都可以做到嵌套,这样子就可以形成相对比较复杂的公式。
\begin{codeshow}
\[
\frac{-b\pm \sqrt{b^{2}-4ac}}{2a}\\
\lim\limits_{x\to 0}\frac
{x\cdot \frac{\cos x -1}{\cos x}}{x^{3}}
\]
\end{codeshow}

除了在分式中会经常用到嵌套以外,矩阵里这种情况也很常见,比如分块矩阵,举个例子。当然,我们也可以把零弄大一点,我们只需要将0修改为\lstinline|\text{\large{0}}|就好
\begin{codeshow}
\[
A=\begin{pmatrix}
\begin{matrix}
1 & 0 \\
0 & 1
\end{matrix} & 0 \\
\text{\large{0}} & \begin{matrix}
1 & 0 \\
0 & 1
\end{matrix}
\end{pmatrix}
\]
\end{codeshow}

\subsection{定界符}
嵌套多了式子会变得非常复杂,也就会变得越来越大!可是这个时候如果你使用括号你会发现,它的大小并没有什么变化,这就显得非常的low,影响美观,因此我们会在括号外加一个left或者是right进行大小的控制。举例如下。
\begin{codeshow}
\[
\lim\limits_{x\to 0}\left(\frac
{a^{x}+b^{x}+c^{x}}{3}\right)
^{\tfrac{1}{x}}
\]
\end{codeshow}

\begin{codeshow}
\[ \Bigg< \bigg\{ \Big[ \big( xyz \big) \Big] \bigg\} \Bigg> \]
\end{codeshow}

学了定界符之后,就可以完全实现矩阵的部分形态了,比方说排版一个增广矩阵。

\begin{codeshow}
\[
\left(
\begin{tabular}{ccc|c}
1 & 1 & 1 & 1 \\
1 & 1 & 1 & 1 \\
1 & 1 & 1 & 1 
\end{tabular}
\right)
\]
\end{codeshow}

定界符必须成对出现,对\textbf{公式组}而言,定界符需要用在公式组环境(align、alignat、gather、aligned、alignedat、gathered)外面。另外,定界符必须成对出现,没有定界符的一侧使用\lstinline|\left.|或者\lstinline|\right.|来代替。了解定界符之后,我们就可以利用定界符和公式组环境做出如下排版,和高数书上的公式效果差不多!

\begin{equation}
\left\{
\begin{gathered}
\frac{\partial c_i}{\partial t}+\nabla \cdot (-D_{i} \nabla c_{i})+u \cdot \nabla c_{i}=R_i \\
N_{i}=-D_{i}\nabla c_{i}+uc_{i}
\end{gathered}
\right.
\end{equation}

\begin{latex}
\begin{equation}
\left\{
\begin{gathered}
\frac{\partial c_i}{\partial t}+\nabla \cdot (-D_{i} \nabla c_{i})+u \cdot \nabla c_{i}=R_i \\
N_{i}=-D_{i}\nabla c_{i}+uc_{i}
\end{gathered}
\right.
\end{equation}
\end{latex}

\subsection{数学字体}
标准的\LaTeX 提供的数学字体有以下几种。简单的文档中,这些字体已经够用了,如果要使用更高级的字体,可查阅\CTeX 宏包说明。
\begin{codeshow}
\[
\mathit{ABCDE}\]
\[
\mathrm{ABCDE}\\\]
\[
\mathbf{ABCDE}\\\]
\[
\mathsf{ABCDE}\\\]
\[
\mathtt{ABCDE}\\\]
\end{codeshow}

\subsection{希腊字母}
有时间排个表在这,不着急。

\subsection{符号}
规范的函数符号输入是使用命令来输入,比如指数、对数以及简单的三角函数符号等。

\begin{codeshow}
$ \sin~\cos~\exp~\log $\\
\[\mathcal{L}(X_i|\lambda) = \sum\limits_{j=1}^{n_i}\log p(x_{ij}|\lambda) \]
\end{codeshow}

调用amsmath宏包后,大多数函数符号能够使用反斜杠加名称直接打出,例如:

\begin{codeshow}
$ \sin \quad \ln \quad \arccos $
\end{codeshow}

像$ \arcsec~\arccot~\arccsc $这三个函数,amsmath宏包就没有定义,这就需要我们自己定义这样一个新的函数命令。
\begin{latex}
%定义一些amsmath没有定义的函数
\DeclareMathOperator{\arcsec}{arcsec}
\DeclareMathOperator{\arccot}{arccot}
\DeclareMathOperator{\arccsc}{arccsc}
\end{latex}

\bz{关于数学环境中的省略号}

\lstinline|\ldots|是\textbf{列举}中用的省略号,而\lstinline|\cdots|是\textbf{运算(连加、连乘)}中用的省略号,二者主要区别在于位置一高一低,切勿混用。

\begin{codeshow}
\begin{gather*}
\{ (X_1,y_1), \ldots, (X_i,y_i), \ldots, (X_{N_{B}},y_{N_{B}}) \}\\
N_I=n_1 +n_2 + \cdots +n_T
\end{gather*}
\end{codeshow}

\subsection{转置符号}
转置符号并没有严格的规定,好几种都在普遍被使用。但是有一点是明确的,转置符号不能是斜体。常见的转置符号大概有四种。

\begin{codeshow}
$\mathbf{A}^\mathrm{T}$\\
$\mathbf{A}^\top$\\
$\mathbf{A}^\mathsf{T}$\\
$\mathbf{A}^\intercal$
\end{codeshow}

推荐使用第三或者第四个,其中\lstinline|\intercal|符号需要使用\hologo{AmSTeX}宏包。


\section{公式环境}

\subsection{单行公式equation}

无论公式多长,都被排版成一行,并给出一个序号。其间,换行命令无效,换段非法并会报错。

\begin{codeshow}
\begin{equation}
\frac{\partial c_i}{\partial t}+\nabla \cdot (-D_{i} \nabla c_{i})+u \cdot \nabla c_{i}=R_i
\end{equation}
\end{codeshow}



\subsection{公式组align和alignat}

该环境可以使公式组或者多行公式关于某个字符对齐,\lstinline|\\|换行,\& 用于分列,\textbf{奇数列会右对齐,偶数列会左对齐},公式组的每一行都会有一个编号。

\begin{latex}
\begin{align}
&\lim\limits_{x\to 1}\left(\frac{1}{1-x}-\frac{3}{1-x^3}\right)\\
= &\lim\limits_{x\to 1}\left(\frac{x^2+x-2}{1-x^3}\right)  \\
= & \lim\limits_{x\to 1}\frac{(x+2)(x-1)}{(1-x)(x^2+x+1)}\\
= & \lim\limits_{x\to 1}\frac{-(x+2)}{x^2+x+1}\\
= & -1
\end{align}
\end{latex}

\begin{align}\label{align-lim}
&\lim\limits_{x\to 1}\left(\frac{1}{1-x}-\frac{3}{1-x^3}\right)\\
= &\lim\limits_{x\to 1}\left(\frac{x^2+x-2}{1-x^3}\right)  \\
= & \lim\limits_{x\to 1}\frac{(x+2)(x-1)}{(1-x)(x^2+x+1)}\\
= & \lim\limits_{x\to 1}\frac{-(x+2)}{x^2+x+1}\\
= & -1
\end{align}

\begin{latex}
\begin{align}
A_{1} & = B_{1}B_{2} & A_{2} & =B_{2} \\
A_{3} & = B_{3} & A_{3}A_{4} & =B_{4}
\end{align}
\end{latex}

\begin{align}
A_{1} & = B_{1}B_{2} & A_{2} & =B_{2} \\
A_{3} & = B_{3} & A_{3}A_{4} & =B_{4}
\end{align}

\note{列对之间的空白与列对两侧的空白相等,像这种公式比较短的情况下就很丑。这时可以使用alignat环境手动控制公式间的空白,该环境必须在参数中指定列队个数。}

\begin{latex}
\begin{alignat}{2}
A_{1} & = B_{1}B_{2} \quad & A_{2} & =B_{2} \\
A_{3} & = B_{3} & A_{3}A_{4} & =B_{4}
\end{alignat}
\end{latex}

\begin{alignat}{2}
A_{1} & = B_{1}B_{2} \quad & A_{2} & =B_{2} \\
A_{3} & = B_{3} & A_{3}A_{4} & =B_{4}
\end{alignat}

\subsection{公式组gather}

用于编写中心对称的公式组,以\lstinline|\\|换行以区分每个公式,每个公式都会被编号。

\begin{latex}
\begin{gather}
\frac{\partial c_i}{\partial t}+\nabla \cdot (-D_{i} \nabla c_{i})+u \cdot \nabla c_{i}=R_i \\
N_{i}=-D_{i}\nabla c_{i}+uc_{i}
\end{gather}
\end{latex}

\begin{gather}
\frac{\partial c_i}{\partial t}+\nabla \cdot (-D_{i} \nabla c_{i})+u \cdot \nabla c_{i}=R_i \\
N_{i}=-D_{i}\nabla c_{i}+uc_{i}
\end{gather}

\subsection{多行公式multline}

适用于长公式在公式中间直接换行的情况,长公式换行并无规矩,通常在关系符(如=)和二元符之后换行。

\begin{latex}
\begin{multline}
\frac{\rho}{\epsilon_p} \left(\frac{\partial u}{\partial t}+(u \cdot \nabla)\frac{u}{\epsilon_p}\right)=\\
\nabla \cdot \left[ -pl+\frac{\mu}{\epsilon_p}\left( \nabla u+(\nabla u)^T\right)-\frac{2\mu}{3\epsilon_p}(\nabla \cdot u)l\right]-\left( \mu \kappa^{-1} + \beta_{F}u+ \frac{Q_{br}}{\epsilon^{2}_{p}} \right)u+F
\end{multline}
\end{latex}

\begin{multline}
\frac{\rho}{\epsilon_p} \left(\frac{\partial u}{\partial t}+(u \cdot \nabla)\frac{u}{\epsilon_p}\right)=\\
\nabla \cdot \left[ -pl+\frac{\mu}{\epsilon_p}\left( \nabla u+(\nabla u)^T\right)-\frac{2\mu}{3\epsilon_p}(\nabla \cdot u)l\right]-\left( \mu \kappa^{-1} + \beta_{F}u+ \frac{Q_{br}}{\epsilon^{2}_{p}} \right)u+F
\end{multline}

\subsection{多行公式split}

适用于关于某个符号对齐的长公式,例如,我们将式\eqref{align-lim}用split环境排版,用equation环境赋予其编号,整个公式只会得到一个编号,更符合排版规范。该环境以\&分列,至多两列,以\lstinline|\\|换行。

\note{split环境不能产生编号,需要外在的公式环境提供;split环境不能与multline嵌套;autoref可以生成“式1.1”,eqref可以生成“(1.1)”,视情况使用,展示无法做到直接引用成“式(1.1)”}

\begin{latex}
\begin{equation}
\begin{split}
&\lim\limits_{x\to 1}\left(\frac{1}{1-x}-\frac{3}{1-x^3}\right)\\
= &\lim\limits_{x\to 1}\left(\frac{x^2+x-2}{1-x^3}\right)  \\
= & \lim\limits_{x\to 1}\frac{(x+2)(x-1)}{(1-x)(x^2+x+1)}\\
= & \lim\limits_{x\to 1}\frac{-(x+2)}{x^2+x+1}\\
= & -1
\end{split}
\end{equation}
\end{latex}

\begin{equation}
\begin{split}
&\lim\limits_{x\to 1}\left(\frac{1}{1-x}-\frac{3}{1-x^3}\right)\\
= &\lim\limits_{x\to 1}\left(\frac{x^2+x-2}{1-x^3}\right)  \\
= & \lim\limits_{x\to 1}\frac{(x+2)(x-1)}{(1-x)(x^2+x+1)}\\
= & \lim\limits_{x\to 1}\frac{-(x+2)}{x^2+x+1}\\
= & -1
\end{split}
\end{equation}

\subsection{breqn宏包}
冲突太多,有待测试。

\subsection{公式块}

align(alignat)、gather产生的公式组只能出现在行间,无法做为一个块出现在行内,而很多情况下我们需要一个公式组做为块出现在行间。这时我们可以使用\textbf{aligned(alignedat)、gathered}公式块环境来完成,每一行可以放置多个公式块,但块环境不提供编号。

\begin{codeshow}
\begin{equation}
\begin{aligned}
f(x,y) & =0 \\
z & =c
\end{aligned}
\quad \text{以及} \quad
\begin{gathered}
x=t\cos t \\
z=at
\end{gathered}
\end{equation}
\end{codeshow}

我们可以用公式块做一些怪东西,比如让
$\begin{aligned}f(x,y) & =0 \\z & =c\end{aligned}$
~以及~
$\begin{gathered}x=t\cos t \\z=at\end{gathered}$
做为行间公式出现在一行。让人想起了高数书上的排版呢!!需要注意的是\textbf{公式块环境只能用在数学环境中},实现的代码如下。

\begin{latex}
$\begin{aligned}f(x,y) & =0 \\z & =c\end{aligned}$
~以及~
$\begin{gathered}x=t\cos t \\z=at\end{gathered}$
\end{latex}

\section{矩阵环境}
矩阵的环境和表格有点相似,所以用法也和列表几乎相同,举个最简单的矩阵例子。
\begin{codeshow}
\[
A=\begin{matrix}
a_{11} & a_{12} & a_{13}\\
0 & 0 & 0\\
0 & 0 & 0
\end{matrix}
\]
\end{codeshow}

那我要写带括号的呢?没关系,不同的矩阵环境会形成不同的括号。这里的matrix就不形成括号,pmatrix形成小括号,bmatrix形成中括号,vmatrix形成竖线(行列式形式),Bmatrix形成大括号,Vmatrix形成双竖线。
\begin{codeshow}
\begin{gather*}
    %居中的公式组环境,不编号
    \begin{pmatrix}1 & 2\\
    3 & 4\end{pmatrix}
    \begin{bmatrix}1 & 2\\
    3 & 4\end{bmatrix}
    \begin{vmatrix}1 & 2\\
    3 & 4\end{vmatrix}\\
    \begin{Bmatrix}1 & 2\\
    3 & 4\end{Bmatrix}
    \begin{Vmatrix}1 & 2\\
    3 & 4\end{Vmatrix}
\end{gather*}
\end{codeshow}

矩阵的元素有时候会很多,需要使用省略号去忽略,而省略号在tex中有专门的命令,列举如下。
\begin{codeshow}
\[
\ldots \cdots \vdots \ddots \dotsc
\]
\end{codeshow}

\begin{codeshow}
\[\begin{bmatrix}
 1      & 2      & \cdots & 4      \\
 7      & 6      & \cdots & 5      \\
 \vdots & \vdots & \ddots & \vdots \\
 8      & 9      & \cdots & 0      \\
\end{bmatrix}\]
\end{codeshow}


\section{定理环境}